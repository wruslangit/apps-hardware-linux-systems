%% =========================================
%% File: inc: Table-Terminology-1of2.tex 
%% =========================================

\begin{table}[ht]
	\begin{center}
		\begin{tabular}{ |p{0.5cm}|p{3.4cm}|p{11.0cm}| }
			\rowcolor{LIGHTCYAN}			
			\hline \multicolumn{3}{|c|}{\textbf{CNC Terminology Part 1 of 2}} \\ [1.0ex]
			
			\hline 1,2  & S/W	and H/W	      & Software and Hardware\\ 
	
			\hline 3 & CNC G-Code File & The S/W file that is generated by a CAM application (including CAM interpolation) based on model or drawing files like DXF, STL (generated from CAD) and so on. This G-Code file, like RS-274D or STEP-NC file, contains instructions like various tool motion commands (G-group), machine service commands (M-group), and so on.\\ 
			
			\hline 4 & Command line  & A single G-Code file S/W instruction. The contents of a CNC G-Code file is an ordered list of CNC command lines.\\ 
			
			\hline 5 & CNC G-Code Interpreter & The S/W application or program component that interprets G-code command lines and segregates into various groups like motion command G-group, services command M-group, and so on.\\ 
			
			\hline 6 & CNC G-Code Interpolator  & The S/W application or program component that converts the G-Code motion command lines into signal commands and save them in a CNC Signals file.\\ 
			
			\hline 7 & Signal command & A single signal file S/W instruction. \\
			\hline 8 & CNC Signal file & The contents of a CNC Signal file is an ordered list of CNC signal commands. The term reference signal used in most literature is synonymous to this signal command. A reference signal basically refers to the signal after CNC interpretation and interpolation of a single G-Code line command (G-group, M-Group, etc.). \\  
			
			\hline 9 & CNC Signal Driver & A S/W application or program component that reads the CNC Signal file and interprets a signal command. This application then generates the appropriate H/W electrical pulses using a specific hardware device. The pulses are finally sent to its destination, for example, the servo-driver and servo-motor hardware pair. \\ 
			
			\hline 10 & CNC Conversion from software to hardware electrical pulses & This conversion is the task of the CNC Signal Driver software. It is the boundary or interface point in CNC machine operations, where software codes generate  actual physical electrical pulses. The electrical pulses, for example, drive the servo-driver and servo-motor pair, meaning, ultimately driving the CNC machine. The CNC hardware executes only on recognizing electrical pulses. The software codes are not of concern to the CNC machine.\\ 

			\hline
		\end{tabular}
		\caption{CNC Terminology Part 1 of 2}		
		\label{table:CNC-Terminology Part 1 of 2}
	\end{center}
\end{table}  
