\pagebreak
%% ===========================================
\begin{tcolorbox}

\section{zzAdvice-on-Indexing}





\textbf{INDEXING HOW TO Guidelines - TO REMOVE LATER}	
% \vspace*{1\baselineskip}

% ==========================================================
\lstset{basicstyle=\footnotesize, numberstyle=\tiny\color{blue}, frame=single, numbers=left, firstnumber=1, stepnumber=1, numbersep=1pt, xleftmargin=2.0em, framexleftmargin=1.5em, xrightmargin=0.0em, breaklines=true, breakatwhitespace=false, breakindent=5pt, prebreak=\space, postbreak=\space }
% ==========================================================
\begin{lstlisting}[caption={zzAdvice-on-Indexing}, label=zzAdvice-on-Indexing]

% (1) INFO FOR APPENDICES

Since glossaries-extra internally loads the glossaries package, you also need to have glossaries installed and all the packages that glossaries depends on (including, but not limited to, tracklang, mfirstuc, etoolbox, xkeyval (at least version dated 2006/11/18), textcase, xfor, datatool-base and amsgen

REF: glossaries-extra Manual
This glossary style was setup using:

\usepackage[xindy,
	nonumberlist,
	toc,
	nopostdot,
	style=altlist,
	nogroupskip]{glossaries}

bib2gls = An indexing application that combines two functions in one: 

	(1) fetches entry definition from a .bib file based on information provided in the .aux file (similar to bibtex); 
	
	(2) hierarchically sorts and collates location lists (similar to makeindex and xindy). This application is designed for use with glossaries-extra and can not be used with just the base glossaries package.



% (2) PLACED INSIDE PREAMBLE SECTION


% (3) PLACED INSIDE DOCUMENT SECTION


% (4) EXAMPLES

	\section{Index - TO DO}
	
	idx = index file\\
	ist = file\\
	acr = file\\
	acn = acronym file\\
	gls = glossary file\\
	glo = glossary file\\
	Command line shell script (/bin/bash). \\
	
	%% #!/bin/bash\\
	makeindex MainRoot-doc.idx\\
	makeindex -s MainRoot-doc.ist -o MainRoot-doc.acr MainRoot-doc.acn\\
	makeindex -s MainRoot-doc.ist -o MainRoot-doc.gls MainRoot-doc.glo\\
	pdflatex MainRoot-doc.tex\\
	pdflatex MainRoot-doc.tex\\
	%% echo "Alhamdulillah 3 times WRY. \n"

...
...
\end{lstlisting}
\end{tcolorbox}
